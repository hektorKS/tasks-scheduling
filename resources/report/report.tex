\documentclass[12pt,a4paper]{article}

\usepackage[T1]{fontenc}
\usepackage[polish]{babel}
\usepackage[utf8]{inputenc}
\usepackage{lmodern}
\usepackage{hyperref}
\usepackage{graphicx}
\usepackage{enumerate}
\usepackage{enumitem}

\let\lll\undefined
\usepackage{amssymb}
\newcommand{\lsim}{\lesssim}

\usepackage{mathtools}
\DeclarePairedDelimiter{\ceil}{\lceil}{\rceil}

\setlength\parindent{0pt}

\usepackage{geometry}
\newgeometry{tmargin=3.5cm, bmargin=3.5cm, lmargin=3cm, rmargin=3cm}

\usepackage{array,multirow}

\selectlanguage{polish}

\title{Praktyka i teoria szeregowania zadań 2018/2019\\ Projekt 1 - wyniki pomiarów}

\author{Konrad Szymański 127240 I4}

\date{\vfill \today}
\begin{document}
\maketitle
\newpage

\section{n = 10, h = 0.2}
\begin{table}[!h]
 \begin{tabular}{|c|c|c|c|c|} 
	\hline
	 k&Upper bound( R )&Funkcja celu( K )&Błąd względny( (K-R)/R*100\% )&Czas\\ \hline
	1&1.936&1936&0.0\%&0.0001s\\ \hline
	2&1.042&1077&3.36\%&0.00012s\\ \hline
	3&1.586&1646&3.78\%&0.00011s\\ \hline
	4&2.139&2183&2.06\%&0.00013s\\ \hline
	5&1.187&1211&2.02\%&0.00014s\\ \hline
	6&1.521&1548&1.78\%&0.00011s\\ \hline
	7&2.170&2349&8.25\%&0.00012s\\ \hline
	8&1.720&1734&0.81\%&0.00011s\\ \hline
	9&1.574&1594&1.27\%&0.0001s\\ \hline
	10&1.869&1885&0.86\%&0.00012s\\ \hline
 \end{tabular}
 	\centering
\end{table}

\section{n = 100, h = 0.4}
\begin{table}[!h]
 \begin{tabular}{|c|c|c|c|c|} 
	\hline
	 k&Upper bound( R )&Funkcja celu( K )&Błąd względny( (K-R)/R*100\% )&Czas\\ \hline
	1&89.588&87144&-2.73\%&0.01669s\\ \hline
	2&74.854&73977&-1.17\%&0.01604s\\ \hline
	3&85.363&79712&-6.62\%&0.01656s\\ \hline
	4&87.730&80585&-8.14\%&0.01812s\\ \hline
	5&76.424&71630&-6.27\%&0.01554s\\ \hline
	6&86.724&78708&-9.24\%&0.01435s\\ \hline
	7&79.854&80036&0.23\%&0.01399s\\ \hline
	8&95.361&94752&-0.64\%&0.01612s\\ \hline
	9&73.605&71071&-3.44\%&0.01464s\\ \hline
	10&72.399&73066&0.92\%&0.01452s\\ \hline
 \end{tabular}
 	\centering
\end{table}

\newpage

\section{n = 500, h = 0.6}
\begin{table}[!h]
 \begin{tabular}{|c|c|c|c|c|} 
	\hline
	 k&Upper bound( R )&Funkcja celu( K )&Błąd względny( (K-R)/R*100\% )&Czas\\ \hline
	1&1.581.233&1.644.084&3.97\%&0.55686s\\ \hline
	2&1.715.332&1.764.970&2.89\%&0.57952s\\ \hline
	3&1.644.947&1.706.825&3.76\%&0.57328s\\ \hline
	4&1.640.942&1.699.461&3.57\%&0.59728s\\ \hline
	5&1.468.325&1.488.165&1.35\%&0.54479s\\ \hline
	6&1.413.345&1.467.991&3.87\%&0.57979s\\ \hline
	7&1.634.912&1.676.834&2.56\%&0.56406s\\ \hline
	8&1.542.090&1.578.410&2.36\%&0.58865s\\ \hline
	9&1.684.055&1.721.468&2.22\%&0.57299s\\ \hline
	10&1.520.515&1.554.160&2.21\%&0.59827s\\ \hline
 \end{tabular}
 	\centering
\end{table}

\section{n = 1000, h = 0.8}
\begin{table}[!h]
 \begin{tabular}{|c|c|c|c|c|} 
	\hline
	 k&Upper bound( R )&Funkcja celu( K )&Błąd względny( (K-R)/R*100\% )&Czas\\ \hline
	1&6.411.581&8.057.467&25.67\%&2.99977s\\ \hline
	2&6.112.598&7.936.730&29.84\%&2.82845s\\ \hline
	3&5.985.538&8.053.593&34.55\%&3.10222s\\ \hline
	4&6.096.729&8.124.088&33.25\%&2.85385s\\ \hline
	5&6.348.242&8.548.800&34.66\%&3.00863s\\ \hline
	6&6.082.142&8.323.781&36.86\%&2.90463s\\ \hline
	7&6.575.879&8.512.139&29.44\%&2.95387s\\ \hline
	8&6.069.658&8.113.120&33.67\%&2.8604s\\ \hline
	9&6.188.416&8.167.610&31.98\%&2.91793s\\ \hline
	10&6.147.295&8.100.751&31.78\%&3.0959s\\ \hline
 \end{tabular}
 	\centering
\end{table}

\end{document}

